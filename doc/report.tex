\documentstyle[]{article}
\newcommand{\SingleSpace}{\edef\baselinestretch{0.9}\Huge\normalsize}
\newcommand{\DoubleSpace}{\edef\baselinestretch{1.4}\Huge\normalsize}
\newcommand{\QuadSpace}{\edef\baselinestretch{2.2}\Huge\normalsize}
\newcommand{\dlxsim}{{\bf DLXsim}}
\newcommand{\code}[1]{{\sf #1}}
\newcommand{\dlxcom}[1]{{\bf (dlxsim) {\tt #1}}}
\newcommand{\instr}[4]{#1 & {\bf #2} & #3 & #4 \\}
\newenvironment{assembly}{\begin{tabular}{llll}}{\end{tabular}}
\newenvironment{mylist}{\begin{list}{}{\leftmargin .6in \labelwidth .5in \labelsep .1in \itemsep .1in}}{\end{list}{}{}}
\newenvironment{commandlevel}{\begin{list}{}{}}{\end{list}{}{}}
\newenvironment{summary}[1]{\if@twocolumn
\section*{#1} \else
\begin{center}
{\bf #1\vspace{-.5em}\vspace{0pt}}
\end{center}
\quotation
\fi}{\if@twocolumn\else\endquotation\fi}
% % Page format %
% \edef\baselinestretch{1.25}
\setlength{\oddsidemargin}{0in}
\setlength{\evensidemargin}{0in}
\setlength{\headsep}{18pt}
\setlength{\topmargin}{0pt} 
\setlength{\textheight}{8.7in}
\setlength{\textwidth}{6.5in}
\begin{document}
% % Title Page %
\title{DLXsim -- A Simulator for DLX}
\author{Larry B. Hostetler \and Brian Mirtich}
\maketitle

\section{Introduction}

Our project involved writing a simulator ({\dlxsim}) for the DLX
instruction set (as described in {\em Computer Architecture, A
Quantitative Approach} by Hennessy and Patterson).
\dlxsim\ is an interactive program that loads DLX assembly programs
and simulates the operation of a DLX computer on those programs,
allowing both single-stepping and continuous execution through the DLX
code.  \dlxsim\ also provides the user with commands to set
breakpoints, view and modify memory and registers, and print statistics on the
execution of the program allowing the user to collect various
information on the run-time properties of a program.  We expect that a
major use for this tool will be in association with future CS 252
classes to aid in the understanding of this instruction set.

A complete overview of the interface provided by the simulator can be
found in the user manual for \dlxsim, which has been included after
this section.  Later in this paper, a few sample runs of the simulator
will also be given.

We decided that since the MIPS instruction set has many similarities
with DLX, and a good MIPS simulator (available from Ousterhaut)
already exists, it would be a better use of our time to modify that
simulator to handle the DLX description.  This simulator was built on
top of the Tcl interface, providing a programming type environment for
the user as well.

The main problem we encountered when rewriting the simulator was that
there are a couple of fundamental differences between the DLX and MIPS
architectures.  Following is a list of the main differences we
identified between the two architectures.

\begin{itemize}

\item In MIPS, branch and jump offsets are stored as the number of
words, where DLX stores the number of bytes.  This has the effect of
allowing jumps on MIPS to go four times as far.

\item MIPS jumps have a non-obvious approach to determining the
destination address: the bits in the offset part of the instruction
simply replace the lower bits in the program counter.  DLX chooses a
more conventional approach in that the offset is sign extended, and
then added to the program counter.

\item In the MIPS architecture, conditional branches are based on the
result of a comparison between any two registers.  DLX has only two
main conditional branch operations which branch on whether a register
is zero or non-zero.

\item DLX provides load interlocks, while the MIPS 2000 does not.

\item MIPS 2000 provides instructions for unaligned accesses to
memory, while DLX does not.

\item The result of a MIPS multiply or divide ends up in two special
registers (HI and LO) allowing 64 bit results; the result of a DLX
multiply is placed in the chosen general purpose register, and must
therefore fit into 32 bits.

\end{itemize}

Because of the large number of similarities between DLX and MIPS, we
based our opcodes on those used by the MIPS machine (where MIPS had
equivalent instructions).  Where DLX had instructions with no MIPS
equivalent, we grouped such similar DLX instructions and assigned to
them blocks of unused opcodes.  Below, you will find the opcode
numbers used for the DLX instructions.  Register-register
instructions have the {\bf special} opcode, and the instruction is
specified in the lower six bits of the instruction word.  Similarly,
floating point instructions have the {\bf fparith} opcode, and the
actual instruction is again found in the lower six bits of the word.

\vspace{.25in}

\vbox{
\begin{center}
Main opcodes
\end{center}

\begin{tabular}{r|c|c|c|c|c|c|c|c|}
 & \$00 & \$01 & \$02 & \$03 & \$04 & \$05 & \$06 & \$07 \\ \hline
\$00 & SPECIAL & FPARITH & J & JAL & BEQZ & BNEZ & BFPT & BFPF \\ \hline
\$08 & ADDI & ADDUI & SUBI & SUBUI & ANDI & ORI & XORI & LHI \\ \hline
\$10 & RFE & TRAP & JR & JALR & & & & \\ \hline
\$18 & SEQI & SNEI & SLTI & SGTI & SLEI & SGEI & & \\ \hline
\$20 & LB & LH & & LW & LBU & LHU & LF & LD \\ \hline
\$28 & SB & SH & & SW & & & SF & SD \\ \hline
\end{tabular}
}

\vspace{.25in}

\vbox{
\begin{center}
Special opcodes (Main opcode = \$00)
\end{center}

\begin{tabular}{r|c|c|c|c|c|c|c|c|}
 & \$00 & \$01 & \$02 & \$03 & \$04 & \$05 & \$06 & \$07 \\ \hline
\$00 & SLLI & & SRLI & SRAI & SLL & & SRL & SRA \\ \hline
\$08 & & & & & TRAP & & & \\ \hline
\$10 & & & & & & & & \\ \hline
\$18 & & & & & & & & \\ \hline
\$20 & ADD & ADDU & SUB & SUBU & AND & OR & XOR & \\ \hline
\$28 & SEQ & SNE & SLT & SGT & SLE & SGE & & \\ \hline
\$30 & MOVI2S & MOVS2I & MOVF & MOVD & MOVFP2I & MOVI2FP & & \\ \hline
\end{tabular}
}

\vspace{.25in}

\vbox{
\begin{center}
Floating Point opcodes (Main opcode = \$01)
\end{center}

\begin{tabular}{r|c|c|c|c|c|c|c|c|}
 & \$00 & \$01 & \$02 & \$03 & \$04 & \$05 & \$06 & \$07 \\ \hline
\$00 & ADDF & SUBF & MULTF & DIVF & ADDD & SUBD & MULTD & DIVD \\ \hline
\$08 & CVTF2D & CVTF2I & CVTD2F & CVTD2I & CVTI2F & CVTI2D & MULT & DIV \\ \hline
\$10 & EQF & NEF & LTF & GTF & LEF & GEF & MULTU & DIVU \\ \hline
\$18 & EQD & NED & LTD & GTD & LED & GED & & \\ \hline
\end{tabular}
}

\vspace{.5in}

The manual entry for \dlxsim\ follows.

\newpage

\markboth{DLXSIM \hfill User Commands \hfill Page\ }{DLXSIM \hfill User Commands \hfill Page\ }
\pagestyle{myheadings}

\begin{mylist}
\item[{\bf NAME}]
\nopagebreak \hfill \\
\dlxsim\ - Simulator and debugger for DLX assembly programs

\item[{\bf SYNOPSIS}]
\nopagebreak \hfill \\
{\bf dlxsim}

\item[{\bf OPTIONS}]
\nopagebreak \hfill \\
{[}-al\#{]} {[}-au\#{]} {[}-dl\#{]} {[}-du\#{]} {[}-ml\#{]} {[}-mu\#{]}
	\begin{mylist}
	\item[{\bf -al\#} \hfill]Select the latency for a floating point add (in clocks).
	\item[{\bf -au\#} \hfill]Select the number of floating point add units.
	\item[{\bf -dl\#} \hfill]Select the latency for a floating point divide.
	\item[{\bf -du\#} \hfill]Select the number of floating point divide units.
	\item[{\bf -ml\#} \hfill]Select the latency for a floating point multiply.
	\item[{\bf -mu\#} \hfill]Select the number of floating point multiply units.
	\end{mylist}

\item[{\bf DESCRIPTION}]
\nopagebreak \hfill \\
\dlxsim\ is an interactive program that loads DLX assembly programs and
simulates the operation of a DLX computer on those programs.  When
\dlxsim\ starts up, it looks for a file named {\bf .dlxsim} in the
user's home directory.  If such a file exists, \dlxsim\ reads it and
processes it as a command file.  \dlxsim\ also checks for a {\bf
.dlxsim} file in the current directory, and executes the commands in
it if the file exists.  Finally, \dlxsim\ loops forever reading commands
from standard input and printing results on standard output.

\item[{\bf NUMBERS}]
\nopagebreak \hfill \\
Whenever \dlxsim\ reads a number, it will accept the number in either
decimal notation, hexadecimal notation if the first two characters of
the number are {\bf 0x} (e.g. 0x3acf), or octal notation if the first
character is {\bf 0} (e.g. 0342).  Two \dlxsim\ commands accept only
floating pointer numbers from the user; these are {\bf fget} and {\bf
fput} and will be described later.

\item[{\bf ADDRESS EXPRESSIONS}]
\nopagebreak \hfill \\
Many of {\dlxsim}'s commands take as input an expression identifying a
register or memory location.  Such values are indicated with the term
{\em address} in the command descriptions below.  Where register names
are acceptable, any of the names {\bf r0} through {\bf r31} and {\bf
f0} through {\bf f31} may be used.  The names {\bf \$0} through {\bf
\$31} may also be used (instead of {\bf r0} through {\bf r31}), but the
dollar signs are likely to cause confusion with Tcl variables, so it
is safer to use {\bf r} instead of {\bf \$}.  The name {\bf pc} may be
used to refer to the program counter.

Symbolic expressions may be used to specify memory addresses.  The
simplest form of such an expression is a number, which is interpreted
as a memory address.  More generally, address expressions may consist of
numbers, symbols (which must be defined in the assembly files
currently loaded), the operators $*$, $/$, $\%$, $+$, $-$, $<<$, $>>$, \&, $|$,
and $\uparrow$ (which have the same meanings and precedences as in C), and
parentheses for grouping.

\item[{\bf COMMANDS}]
\nopagebreak \hfill \\
In addition to all of the built-in Tcl commands, \dlxsim\ provides the following application-specific commands:

\begin{mylist}
\item[{\bf asm} {\em instruction} {[}{\em address}{]}]
\nopagebreak \hfill \\
Treats {\em instruction} as an assembly instruction and returns a
hexadecimal value equivalent to {\em
instruction}.  Some instructions, such as relative branches, will be
assembled differently depending on where in memory the instruction
will be stored.  The {\em address} argument may be used to indicate
where the instruction would be stored; if omitted, it defaults to 0.

\item[{\bf fget} {\em address} {[}{\em flags}{]}]
\nopagebreak \hfill \\
Return the values of one or more memory locations or registers.  {\em
Address} identifies a memory location or register, and {\em flags}, if
present, consists of a number and/or set of letters, all concatenated
together.  If the number is present, it indicates how many consecutive
values to print (the default is 1).  If flag characters are present,
they have the following interpretation:

\begin{mylist}
\item[{\bf d}\hfill] Print values as double precision floating point numbers.
\item[{\bf f}\hfill] Print values as single precision floating point numbers (default).
\end{mylist}

\item[{\bf fput} {\em address number} {[}{\em precision}{]}]
\nopagebreak \hfill \\
Store {\em number} in the register or memory location given by {\em
address}.  If {\em precision} is {\bf d}, the number is stored as a
double precision floating point number (in two words).  If {\em
precision} is {\bf f} or no {\em precision} is given, the number is
stored as a single precision floating point number.

\item[{\bf get} {\em address} {[}{\em flags}{]}]
\nopagebreak \hfill \\
Similar to {\bf fget} above, this command is for all types except
floating point.  If flag characters are present, they have the following
interpretation:

\begin{mylist}
\item[{\bf B}\hfill] Print values in binary.
\item[{\bf b}\hfill] When printing memory locations, treat each byte as
a separate value.
\item[{\bf c}\hfill] Print values as ASCII characters.
\item[{\bf d}\hfill] Print values in decimal.
\item[{\bf h}\hfill] When printing memory locations, treat each halfword
as a separate value.
\item[{\bf i}\hfill] Print values as instructions in the DLX assembly
language.
\item[{\bf s}\hfill] Print values as null-terminated ASCII strings.
\item[{\bf v}\hfill] Instead of printing the value of the memory location
referred to by {\em address}, print the address itself as the value.
\item[{\bf w}\hfill] When printing memory locations, treat each word as a
separate value.
\item[{\bf x}\hfill] Print values in hexadecimal (default).
\end{mylist}

To interpret numbers as single or double precision floating point, use
the {\bf fget} command.

\item[{\bf go} {[}{\em address}{]}]
\nopagebreak \hfill \\
Start simulating the DLX machine.  If {\em address} is given,
execution starts at that memory address.  Otherwise, it continues from
wherever it left off previously.  This command does not complete until
simulated execution stops.  The return value is an information string
about why execution stopped and the current state of the machine.

\item[{\bf load} {\em file file file} \ldots]
\nopagebreak \hfill \\
Read each of the given {\em files}.  Treat them as DLX assembly
language files and load memory as indicated in the files.  Code (text)
is normally loaded starting at address 0x100, but the {\bf codeStart}
variable may be used to set a different starting address.  Data is
normally loaded starting at address 0x1000, but a different starting
address may be specified in the {\bf dataStart} variable.  The return
value is either an empty string or an error message describing
problems in reading the files.  A list of directives that the loader
understands is in a later section of this manual.

\item[{\bf put} {\em address number}]
\nopagebreak \hfill \\
Store {\em number} in the register or memory location given by {\em
address}.  The return value is an empty string.  To store floating
point numbers (single or double precision), use the {\bf fput}
command.

\item[{\bf quit}\hfill]
Exit the simulator.

\item[{\bf stats} {[}{\bf reset}{]} {[}{\bf stalls}{]} {[}{\bf opcount}{]}
      {[}{\bf pending}{]} {[}{\bf branch}{]} {[}{\bf hw}{]} {[}{\bf all}{]}]
\nopagebreak \hfill \\
This command will dump various statistics collected by the simulator
on the DLX code that has been run so far.  Any combination of options
may be selected.  The options and their results are as follows:

\begin{mylist}
\item[{\bf reset} \hfill]
Reset all of the statistics.
\item[{\bf stalls} \hfill]
Show the number of load stalls and stalls while waiting for a floating
point unit to become available or for the result of a previous
operation to become available.
\item[{\bf opcount} \hfill]
Show the number of each operation that has been executed.
\item[{\bf pending} \hfill]
Show all floating point operations currently being handled by the
floating point units as well as what their results will be and where
they will be stored.
\item[{\bf branch} \hfill]
Show the percentage of branches taken and not-taken.
\item[{\bf hw} \hfill]
Show the current hardware setup for the simulated machine.
\item[{\bf all} \hfill]
Equivalent to choosing all options except {\bf reset}.  This is the default.
\end{mylist}

\item[{\bf step} {[}{\em address}{]}]
\nopagebreak \hfill \\
If no {\em address} is given, the {\bf step} command executes a single
instruction, continuing from wherever execution previously stopped.
If {\em address} is given, then the program counter is changed to
point to {\em address}, and a single instruction is executed from
there.  In either case, the return value is an information string
about the state of the machine after the single instruction has been
executed.

\item[{\bf stop} {[}{\em option args}{]}]
\nopagebreak \hfill \\
This command may take any of the forms described below:

\begin{mylist}
\item[{\bf stop}\hfill]
Arrange for execution of DLX code to stop as soon as possible.  If a
simulation isn't in progress then this command has no effect.
This command is most often used in the {\em command}
argument for the {\bf stop at} command.  Returns an empty string.

\item[{\bf stop at} {\em address} {[}{\em command}{]}]
\nopagebreak \hfill \\
Arrange for {\em command} (a \dlxsim\ command string) to be executed
whenever the memory address identified by {\em address} is read,
written, or executed.  If {\em command} is not given, it defaults to
{\bf stop}, so that execution stops whenever {\em address} is
accessed.  A stop applies to the entire word containing {\em address}:
the stop will be triggered whenever any byte of the word is accessed.
Stops are not processed during the {\bf step} commands or the first
instruction executed in a {\bf go} command.  Returns an empty string.

\item[{\bf stop info}]
\nopagebreak \hfill \\
Return information about all stops currently set.

\item[{\bf stop delete} {\em number number number} \ldots]
\nopagebreak \hfill \\
Delete each of the stops identified by the {\em number} arguments.
Each {\em number} should be an identifying number for a stop, as
printed by {\bf stop info}.  Returns an empty string.

\end{mylist}

\item[{\bf trace} {[}{\bf on} {\em file}{]} {[}{\bf off}{]}]
\nopagebreak \hfill \\
This command toggles the writing of memory access information into a
file for use with the dinero utility.  The options and their results
are as follows:

\begin{mylist}
\item[{\bf on} {\em file} \hfill]
Start writing dinero trace information in the named {\em file}.  If
{\em file} already exists, the information will be appended to it.
Reset all of the statistics.
\item[{\bf off} \hfill]
Stop writing dinero trace information.
\end{mylist}

\end{mylist}

\item[{\bf ASSEMBLY FILE FORMAT}]
\nopagebreak \hfill \\
The assembler built into \dlxsim, invoked using the {\bf load}
command, accepts standard format DLX assembly language programs.  The file is expected to contain lines of the following form:

\begin{itemize}
\item Labels are defined by a group of non-blank characters starting
with either a letter, an underscore, or a dollar sign, and followed
immediately by a colon.  They are associated with the address
immediately following the last block of information stored.  This has
the bad effect that if you have code following a label following a
block of data that does not end on a word boundary (multiple of 4),
the label will not point to the first instruction in the code, but
instead to 1 to 3 bytes before (since the address is only rounded when
it is necessary to correctly align data).  This is done so that if a
label is found in the middle of a data section, it will point to the
start of the next section of data without the data having to be
aligned to a word boundary.  The work-around for this is to use the
{\bf .align} (see below) directive before labels that will not be
aligned with the data following them.  Labels can be accessed anywhere
else within that file, and in files loaded after that if the label is
declared as {\bf .global} (see below).

\item Comments are started with a semicolon, and continue to the end of the line.

\item Constants can be entered either with or without a preceding number sign.

\item The format of instructions and their operands are as shown in
the Computer Architecture book.
\end{itemize}

While the assembler is processing an assembly file, the data and
instructions it assembles are placed in memory based on either a text
(code) or data pointer.  Which pointer is used is selected not by the
type of information, but by whether the most recent directive was {\bf
.data} or {\bf .text}.  The program initially loads into the text
segment.

The assembler supports several directives which affect how it loads
the DLX's memory.  These should be entered in the place where you
would normally place the instruction and its arguments.  The
directives currently supported by \dlxsim\ are:

\begin{mylist}
\item[{\bf .align} $n$ \hfill]
Cause the next data/code loaded to be at the next higher address with
the lower $n$ bits zeroed (the next closest address greater than or
equal to the current address that is a multiple of $2^n$).

\item[{\bf .ascii} ``{\em string1}'', ``{\em string2}'', \ldots]
\nopagebreak \hfill \\
Store the {\em strings} listed on the line in memory as a list of
characters.  The strings are not terminated by a 0 byte.

\item[{\bf .asciiz} ``{\em string1}'', ``{\em string2}'', \ldots]
\nopagebreak \hfill \\
Similar to {\bf .ascii}, except each string is followed by a 0 byte
(like C strings).

\item[{\bf .byte} ``{\em byte1}'', ``{\em byte2}'', \ldots]
\nopagebreak \hfill \\
Store the {\em bytes} listed on the line sequentially in memory.

\item[{\bf .data} {[}{\em address}{]}]
\nopagebreak \hfill \\
Cause the following code and data to be stored in the data area.  If
an {\em address} was supplied, the data will be loaded starting at
that address, otherwise, the last value for the data pointer will be
used.  If we were just reading code based on the text (code) pointer,
store that address so that we can continue from there later (on a {\bf
.text} directive).

\item[{\bf .double} {\em number1}, {\em number2}, \ldots]
\nopagebreak \hfill \\
Store the {\em numbers} listed on the line sequentially in memory as
double precision floating point numbers.

\item[{\bf .float} {\em number1}, {\em number2}, \ldots]
\nopagebreak \hfill \\
Store the {\em numbers} listed on the line sequentially in memory as
single precision floating point numbers.

\item[{\bf .global} {\em label}]
\nopagebreak \hfill \\
Make the {\em label} available for reference by code found in files
loaded after this file.

\item[{\bf .space} {\em size}]
\nopagebreak \hfill \\
Move the current storage pointer forward {\em size} bytes (to leave some
empty space in memory).

\item[{\bf .text} {[}{\em address}{]}]
\nopagebreak \hfill \\
Cause the following code and data to be stored in the text (code)
area.  If an {\em address} was supplied, the data will be loaded
starting at that address, otherwise, the last value for the text
pointer will be used.  If we were just reading data based on the data
pointer, store that address so that we can continue from there later
(on a {\bf .data} directive).

\item[{\bf .word} {\em word1}, {\em word2}, \ldots]
\nopagebreak \hfill \\
Store the {\em words} listed on the line sequentially in memory.

\end{mylist}

\item[{\bf C Library Functions}]
\nopagebreak \hfill \\
\dlxsim\ allows the user access to a few simple C library functions
through the use of the {\bf TRAP} operation.  Currently supported
functions are : {\bf open()} (trap~\#1), {\bf close()} (trap~\#2),
{\bf read()} (trap~\#3), {\bf write()} (trap~\#4), {\bf printf()}
(trap~\#5).  When the appropriate trap is invoked, the first argument
should be located in the word starting at the address in r14, with the
following arguments (as seen in a C statement calling the function) in
words above that (r14+4, r14+8, \ldots).  The result from the function
call will be placed in r1 (this means there is currently no support
for library functions that return floating point values).  If a double
precision floating point value is to be passed to a library function,
it will occupy two adjacent words with the lower word containing the
value of the even valued floating point register, and the higher word
containing the value of the odd valued floating point register (F0 in
0(r14), F1 in 4(r14)).

A call to a C Library Function currently only registers as a trap
instruction in the statistics gathered by the simulator, and does
not affect the instructions executed or cycles counted additionally.

{\bf NOTE:} Any memory accessed by a trap function needed to perform
its work is {\bf not} currently placed in the {\bf dinero} trace file
(if one is active).

\item[{\bf VARIABLES}]
\nopagebreak \hfill \\
\dlxsim\ uses or sets the following Tcl variables:

\begin{mylist}
\item[{\bf codeStart}]
\nopagebreak \hfill \\
If this variable exists, it indicates where to start loading code in
{\bf load} commands.

\item[{\bf dataStart}]
\nopagebreak \hfill \\
If this variable exists, it indicates where to start loading data in
{\bf load} commands.

\item[{\bf insCount}]
\nopagebreak \hfill \\
\dlxsim\ uses this variable to keep a running count of the total number
of instructions that have been simulated so far.

\item[{\bf prompt}\hfill]
If this variable exists, it should contain a \dlxsim\ command string.
\dlxsim\ will execute the command in this string before printing each
prompt, and use the result as the prompt string to print.  If this
variable doesn't exist, or if an error occurs in executing its
contents, then the prompt ``(dlxsim)'' is used.
\end{mylist}

\item[{\bf Future Enhancements}]
\nopagebreak \hfill \\
Fixing the label handling in the assembler so that a label is
associated with the next address used in the assembler (not
necessarily the address following the last memory altering line).

Modify the trap handler to note memory accesses to the {\bf dinero}
trace file when appropriate.

\item[{\bf SEE ALSO}]
\nopagebreak \hfill \\
{\em Computer Architecture, A Quantitative Approach}, by~John L.~Hennessy and David A.~Patterson.

\item[{\bf KEYWORDS}]
\nopagebreak \hfill \\
DLX, debug, simulate

\end{mylist}


\newpage

\pagestyle{plain}

\section{Interactive Sessions with \dlxsim}

To illustrate some of the features of \dlxsim, this section describes
two interactive sessions using examples taken from Chapter 6
of {\it Computer Architecture, A Quantitative Approach} by Hennessy and
Patterson.  The programs used are on page 315 and 317.  The ADDD
instructions have been replaced with MULTD instructions, however, to
show the effects of a slightly longer latency.  Also, TRAP instructions
have been added to terminate execution of the programs when simulating.

\subsection{Sample Datafile}

The examples which follow operate on arrays of numbers.  A common datafile
is used for input to the programs.  This datafile is named {\bf fdata.s}
and is shown below:\\

\begin{assembly}
\instr{}{.data}{0}{}
\instr{}{.global}{a}{}
\instr{a:}{.double}{3.14159265358979}{}
\instr{}{.global}{x}{}
\instr{x:}{.double}{1,2,3,4,5,6,7,8,9,10,11,12,13,14,15,16}{}
\instr{}{.double}{17,18,19,20,21,22,23,24,25,26,27}{}
\instr{}{.global}{xtop}{}
\instr{xtop:}{.double}{28}{}
\end{assembly} \\

The {\bf .data} directive specifies that the data should be loaded in
at location 0.  The {\bf .global} directive add the specified labels
to a global symbol table so that other assembly files can access them.
The {\bf .double} directive stores double precision data to memory.

\subsection{First Example}

The first example uses the program at the bottom of page 315 (with the
ADDD replaced by MULTD).  The program is shown below. \\

\begin{assembly}
\instr{}{ld}{f2,a}{}
\instr{}{add}{r1,r0,xtop}{}
\instr{loop:}{ld}{f0,0(r1)}{}
\instr{}{}{}{; load stall occurs here}
\instr{}{multd}{f4,f0,f2}{}
\instr{}{}{}{; 4 FP stalls}
\instr{}{sd}{0(r1),f4}{}
\instr{}{sub}{r1,r1,\#8}{}
\instr{}{bnez}{r1,loop}{}
\instr{}{nop}{}{; branch delay slot}
\instr{}{trap}{\#0}{; terminate simulation}
\end{assembly} \\

The simulator is invoked by typing {\bf dlxsim} at the system prompt. \\

{\tt \% dlxsim} \\

First the datafile is loaded, using the load command: \\

\dlxcom{load fdata.s} \\

Next, the program may be loaded.  The program above was created with an
editor and saved in the file {\bf f1.s}.  It is loaded in the same way
as the datafile. \\

\dlxcom{load f1.s} \\

To verify that the program has been loaded, the {\bf get} command can
be used to examine memory.  The program is loaded at location 256 by
default.  The second parameter to {\bf get} indicates how many words
to dump.  The {\bf i} suffix tells {\bf get} to dump the contents in
instruction format (i.e.  produce a disassembly). \\

\dlxcom{get 256 9i}
\begin{verbatim}
start:	ld f2,a(r0)
start+0x4:	addi r1,r0,0xe0
loop:	ld f0,a(r1)
loop+0x4:	multd f4,f0,f2
loop+0x8:	sd a(r1),f4
loop+0xc:	subi r1,r1,0x8
loop+0x10:	bnez r1,loop
loop+0x14:	nop
loop+0x18:	trap 0x0
\end{verbatim}

To make sure that the statistics are all cleared (as they should be when
\dlxsim\ is first invoked), use the {\bf stats} command with the relevant
parameters: \\

\dlxcom{stats stalls branch pending hw}
\begin{verbatim}
Memory size: 65536 bytes.

Floating Point Hardware Configuration
 1 add/subtract units, latency =  2 cycles
 1 divide units,       latency = 19 cycles
 1 multiply units,     latency =  5 cycles
Load Stalls = 0
Floating Point Stalls = 0

No branch instructions executed.

Pending Floating Point Operations:
none.
\end{verbatim}

The {\bf hw} specifier causes the memory size and floating point hardware
information to be dumped.  The {\bf stalls} specifier causes the total
load stalls and floating point stalls to be displayed.  The {\bf branch}
specifier causes the branch information (taken vs. not taken) to be displayed;
in this case no branches have been executed yet.  Finally, the {\bf pending}
specifier causes the pending operations in the floating point units to
be displayed (none in this case).
Below, the first four instructions are executed using the {\bf step}
command: \\

\dlxcom{step 256}
\begin{verbatim}
stopped after single step, pc = start+0x4: addi r1,r0,0xe0
\end{verbatim}

\dlxcom{step}
\begin{verbatim}
stopped after single step, pc = loop: ld f0,a(r1)
\end{verbatim}

\dlxcom{step}
\begin{verbatim}
stopped after single step, pc = loop+0x4: multd f4,f0,f2
\end{verbatim}

\dlxcom{step}
\begin{verbatim}
stopped after single step, pc = loop+0x8: sd a(r1),f4
\end{verbatim}

The {\bf stats} command can produce some more interesting results 
at this point.\\

\dlxcom{stats stalls pending}
\begin{verbatim}
Load Stalls = 1
Floating Point Stalls = 0

Pending Floating Point Operations:
multiplier   #1 :  will complete in  4 more cycle(s)  87.964594 ==> F4:F5
\end{verbatim}

A load stall occurred between the third and fourth instructions because of the
F0 dependency.  The multiply instruction has issued, and is being
processed in multiplier unit \#1.  It will complete and store the
double precision value 87.96 into F4 and F5 in four more clock cycles.

The double precision value in F4 can be displayed using the {\bf fget}
command with a {\bf d} specifier (for double precision). \\

\dlxcom{fget f4 d}
\begin{verbatim}
f4:	0.000000
\end{verbatim}

As expected, F4 hasn't received its value yet.  Executing one more instruction
will change the statistics: \\

\dlxcom{step}
\begin{verbatim}
stopped after single step, pc = loop+0xc: subi r1,r1,0x8
\end{verbatim}

\dlxcom{stats stalls pending}
\begin{verbatim}
Load Stalls = 1
Floating Point Stalls = 4

Pending Floating Point Operations:
none.
\end{verbatim}

Since the SD instruction used the result from the multiply instruction,
the multiply was completed before the SD was executed.  The four floating
point stalls required for the multiply to complete were recorded as well.
If F4 is examined now, its value after the writeback is displayed. \\

\dlxcom{fget f4 d}
\begin{verbatim}
f4:	87.964594
\end{verbatim}

To execute the program to completion, the {\bf go} command can be used.
When the TRAP instruction is detected, the simulation will stop. \\

\dlxcom{go}
\begin{verbatim}
TRAP #0 received
\end{verbatim}

To view the cumulative stall and branch information, the {\bf stats} command
can be used. \\

\dlxcom{stats stalls branch}
\begin{verbatim}
Load Stalls = 28
Floating Point Stalls = 112

Branches:  total 28, taken 27 (96.43%), untaken 1 (3.57%)
\end{verbatim}

The loop executed 28 times.  There was a single load stall per iteration, 
for a total of 28 load stalls.  There were 4 floating point stalls per
iteration, for a total of 112 floating point stalls.  Finally, the
conditional branch at the bottom of the loop was taken 27 times, and
fell through on the final time.  All these statistics are reflected above.

To verify the program operated properly, the memory locations containing
the original data can be examined with the {\bf fget} command.  The original
data was stored in the 28 double words beginning at location 8.\\

\dlxcom{fget 8 28d}
\begin{verbatim}
x:	3.141593
x+0x8:	6.283185
x+0x10:	9.424778

... etc. ...

x+0xc8:	81.681409
x+0xd0:	84.823002
xtop:	87.964594
\end{verbatim}

As expected, the initial integer values have all been multiplied by $\pi$.

\subsection{Second Example}

The second example is from page 317 of the aforementioned text.  It
demonstrates the effects of unrolling loops when multiple execution units
are available.  The program, which is shown below, performs the same
operations on the list of numbers as the previous example program. \\

\begin{assembly}
\instr{start:}{ld}{f2,a}{}
\instr{}{add}{r1,r0,xtop}{}
\instr{loop:}{ld}{f0,0(r1)}{}
\instr{}{ld}{f6,-8(r1)}{}
\instr{}{ld}{f10,-16(r1)}{}
\instr{}{ld}{f14,-24(r1)}{}
\instr{}{multd}{f4,f0,f2}{}
\instr{}{multd}{f8,f6,f2}{}
\instr{}{multd}{f12,f10,f2}{}
\instr{}{multd}{f16,f14,f2}{}
\instr{}{}{}{; FP stall here}
\instr{}{sd}{0(r1),f4}{}
\instr{}{sd}{-8(r1),f8}{}
\instr{}{sd}{-16(r1),f12}{}
\instr{}{sub}{r1,r1,\#32}{}
\instr{}{bnez}{r1,loop}{}
\instr{}{sd}{8(r1),f16}{; branch delay slot}
\instr{}{trap}{\#0}{}
\end{assembly} \\

To take full advantage of this unwound loop, \dlxsim\ can be invoked with
a command line argument specifying 4 floating point multiply units should
be included in the hardware configuration.\\

{\bf \%} {\tt dlxsim -mu4} \\

\dlxcom{stats hw}
\begin{verbatim}
Memory size: 65536 bytes.

Floating Point Hardware Configuration
 1 add/subtract units, latency =  2 cycles
 1 divide units,       latency = 19 cycles
 4 multiply units,     latency =  5 cycles
\end{verbatim}

After loading the data and program files, the {\bf step} instruction can
be used to execute the first 10 instructions.  At this point, the last
MULTD instruction has just issued.  The {\bf stats} command can display
the stalls and pending operations. \\

\dlxcom{stats stalls pending}
\begin{verbatim}
Load Stalls = 0
Floating Point Stalls = 0

Pending Floating Point Operations:
multiplier   #0 :  will complete in  1 more cycle(s)  87.964594 ==> F4:F5
multiplier   #1 :  will complete in  2 more cycle(s)  84.823002 ==> F8:F9
multiplier   #2 :  will complete in  3 more cycle(s)  81.681409 ==> F12:F13
multiplier   #3 :  will complete in  4 more cycle(s)  78.539816 ==> F16:F17
\end{verbatim}

It is intersting to see what happens after the next instruction is executed. \\

\dlxcom{step}
\begin{verbatim}
stopped after single step, pc = loop+0x24: sd 0xfff8(r1),f8
\end{verbatim}

\dlxcom{stats stalls pending}
\begin{verbatim}
Load Stalls = 0
Floating Point Stalls = 1

Pending Floating Point Operations:
multiplier   #2 :  will complete in  1 more cycle(s)  81.681409 ==> F12:F13
multiplier   #3 :  will complete in  2 more cycle(s)  78.539816 ==> F16:F17
\end{verbatim}

Since the SD instruction was dependent on the first MULTD instruction, a
floating point stall occurred so the MULTD could complete.  This added stall
cycle also caused the second MULTD to also complete.  The MULTDs have
``caught up'' with the SDs, and no more stalls will occur on this 
iteration.  This is the reason loop unrolling works.  To run the program
to completion, the {\bf go} command can be used. \\

\dlxcom{go}
\begin{verbatim}
TRAP #0 received
\end{verbatim}

To dump all the statistics gathered, the {\bf stats} command is used without
any parameters.

\dlxcom{stats}
\begin{verbatim}
Memory size: 65536 bytes.

Floating Point Hardware Configuration
 1 add/subtract units, latency =  2 cycles
 1 divide units,       latency = 19 cycles
 4 multiply units,     latency =  5 cycles
Load Stalls = 0
Floating Point Stalls = 7

Branches:  total 7, taken 6 (85.71%), untaken 1 (14.29%)

Pending Floating Point Operations:
none.
				INTEGER OPERATIONS
				==================

     ADD        0      ADDI        1      ADDU        0     ADDUI        0  
     AND        0      ANDI        0      BEQZ        0      BFPF        0  
    BFPT        0      BNEZ        7       DIV        0      DIVU        0  
       J        0       JAL        0      JALR        0        JR        0  
      LB        0       LBU        0        LD       29        LF        0  
      LH        0       LHI        0       LHU        0        LW        0  
    MOVD        0      MOVF        0   MOVFP2I        0   MOVI2FP        0  
  MOVI2S        0    MOVS2I        0      MULT        0     MULTU        0  
      OR        0       ORI        0       RFE        1        SB        0  
      SD       28       SEQ        0      SEQI        0        SF        0  
     SGE        0      SGEI        0       SGT        0      SGTI        0  
      SH        0       SLE        0      SLEI        0       SLL        0  
SLLI/NOP        0       SLT        0      SLTI        0       SNE        0  
    SNEI        0       SRA        0      SRAI        0       SRL        0  
    SRLI        0       SUB        0      SUBI        7      SUBU        0  
   SUBUI        0        SW        0      TRAP        1       XOR        0  
    XORI        0  
Total integer operations = 74

			FLOATING POINT OPERATIONS
			=========================

    ADDD        0      ADDF        0    CVTD2F        0    CVTD2I        0  
  CVTF2D        0    CVTF2I        0    CVTI2D        0    CVTI2F        0  
    DIVD        0      DIVF        0       EQD        0       EQF        0  
     GED        0       GEF        0       GTD        0       GTF        0  
     LED        0       LEF        0       LTD        0       LTF        0  
   MULTD       28     MULTF        0       NED        0       NEF        0  
    SUBD        0      SUBF        0  
Total floating point operations = 28
Total operations = 102
Total cycles = 109
\end{verbatim}

The dynamic counts for all instructions are shown, as well as the
statistics previously discussed.  The number of load stalls is seven
in this case, compared to 28 in the first example.  This is the result
of unrolling the loop four times and providing four multiply units in
hardware.  An estimate of the clocks per instruction (CPI) can be
obtained by dividing the total cycles (109) by the total operations
(102).

The two examples above give only a flavor for the types of operations which
may be done in \dlxsim.  The possibilities are endless.

\pagebreak

\section{Internal Operation}

Some information concerning how \dlxsim\ operates internally may be
useful to some users, particularly those who wish to modify or enhance
the simulator.  This section provides an overview of
the simulator and a discussion of the underlying data structures used.
{\it This information is not necessary to use \dlxsim.}
All of the code discussed below is contained in the file \code{sim.c}.

\subsection{Instruction Tables}

\dlxsim\ contains four tables which contain information about the DLX
instruction set.  The first is \code{opTable}.  This table contains 64
entries corresponding to the 64 possible values of the opcode field.  Each
entry consists of an instruction-format pair.  For example, the value of
\code{opTable[5]} is \code{\{OP\_BNEZ, IFMT\}} indicating that opcode 5
is a branch not equal to zero instruction, which uses the I-type format.
Several entries in this table have \code{OP\_RES} as the instruction.  These
entries are reserved for future extensions to the DLX instruction set.

The zero opcode indicates a different table should be used to identify
the instruction.  A second table called \code{specialTable} handles
this case.  In this table are all the register-register operations.
The format is not specified explicitly for these instructions (as it
was in \code{opTable}) because they are all R-type format.  These
instructions all contain a zero in the opcode field and a function
encoding in the lower six bits of the instruction word.  There is also
room in this table for expansion by using entries currently containing
\code{OP\_RES}.

An opcode of one indicates a floating point arithmetic operation.  A
third table, \code{FParithTable} handles these instructions.  As with
\code{specialTable}, all instructions in this table have R-type format.
The exact operation is again specified by the lower six bits of the
instruction word, which are used to index into this table.  Currently
32 entries contain \code{OP\_RES} and are available for future expansion
to the floating point instruction set.

The final table is \code{operationNames}.  This table contains a list of all
the integer instruction names followed by the floating point instruction names.
Each group is arranged alphabetically.  These tables are used to print out
the names of the instructions when a dynamic instruction count is requested.

\subsection{Simulator Support Functions}

This subsection describes the various routines which handle simulator
commands and provide support for the main simulator code.  The function
\code{Sim\_Create} initializes a DLX processor structure and is invoked when
\dlxsim\ is first started.  The memory size of the machine along with
the floating point hardware specification (i.e. unit quantities and
latencies) are specified as parameters.

Two functions, \code{statsReset} and \code{Sim\_DumpStats}, process the
{\tt stats} command in \dlxsim.  The former resets all the statistics
to zero, and the latter processes requests for various statistics.  The
statistics currently taken during simulation are:  load stalls, floating
point stalls, dynamic instruction counts, and conditional branch behavior.
In addition, the floating point hardware and pending floating point operations
can also be examined.  See the description of the {\tt stats} command
for more information on how to request and reset the various statistics.

The functions \code{Sim\_GoCmd} and \code{Sim\_StepCmd} process the simulator's
{\tt go} and {\tt step} commands, respectively.  See the description of these
commands for more information on using them.

The functions \code{ReadMem} and \code{WriteMem} provide the interface
between the simulator and the DLX memory structure.  They insure that the
address accessed is valid, which means in must be within the memory's
range and it must be on a word boundary.  Otherwise, appropriate error
handling occurs.

\subsection{Compilation of Instructions}

To improve efficiency, \dlxsim\ ``compiles'' the instructions as it
first encounters them.  To understand how this works, it is necessary
to examine the structure of a single word of the DLX memory.  A single
memory word contains several fields: \code{value}, \code{opCode},
\code{rs1}, \code{rs2}, \code{rd}, and
\code{extra}.  A DLX program to be simulated is written in DLX
assembly language.  Such a program is automatically assembled into
machine code as it is loaded.  The actual machine codes are stored in
the \code{value} fields of the memory words.  The \code{value} field
represents the number actually stored at a particular memory word.
The \code{opCode} field of each memory word is initially set to the
special value \code{OP\_NOT\_COMPILED}.

When the simulator executes an instruction, it first examines the
\code{opCode} field of the memory word pointed to by the program counter.
If this field is a valid opcode (specified in the tables discussed
above), the appropriate action for that instruction occurs.  If the
\code{opCode} field contains the value \code{OP\_NOT\_COMPILED}, the
function \code{Compile} is invoked.  This function looks at the actual
word stored in the \code{value} field.  The bits corresponding to the
opcode and function fields are examined to determine what the
instruction is.  Depending on the instruction type, the two source
register specifiers and destination register specifier may be
extracted and stored in the fields \code{rs1}, \code{rs2}, and
\code{rd}.  If a 16-bit immediate value is present (for I-type
instructions) or a 26-bit offset is present (for J-type instructions),
this value is extracted and stored in the \code{extra} field of the
memory word.  The special code for the instruction is stored in the
\code{opCode} field of the word, which previously contained the value
\code{OP\_NOT\_COMPILED}.  These special codes are not the real DLX
opcodes, but rather the pseudo-opcodes defined in the file
\code{dlx.h}.

When a compiled instruction is subsequently encountered, no shifting
or masking operations are required to access the register specifiers or
immediate values; the required information is already present in the
appropriate fields of the memory words (\code{rs1}, \code{rs2},
\code{rd}, and \code{extra}).  This allows the simulator to execute
much faster.  The actual machine code for the instruction can still be
examined through the \code{value} field, and this is the value printed
when the word is examined with the {\tt get} command.

\subsection{Main Simulation Loop}

\code{Simulate} is the main function of the simulator.  The heart of this
function is basically a very large 
\code{switch} statement, based on the \code{opCode} field of the memory
word pointed to by the program counter.  There is a case for each
integer and floating point instruction.  \code{Simulate} loops through the
basic fetch-decode-execute cycle until a stop command is received or some
other exceptional condition occurs.

\subsubsection{Load Stalls}

DLX has a latency of one cycle on load instructions.  In other words,
the result is not yet present in the destination register on the cycle
immediately following the load instruction.  To address this problem,
DLX has load interlocks which cause the pipeline to stall if an
instruction immediately following a load instruction reads the value
in the load's destination register.  \dlxsim\ records the occurance of
these load stall cycles for statistical purposes.  Several variables
are set during the processing of the following load instructions:
\code{LB}, \code{LBU}, \code{LH}, \code{LHU} \code{LW}, \code{LF},
and \code{LD}.  \code{LHI} is not included since the value to be
loaded is contained in the instruction and there is no extra latency.
For the other load instructions, the destination register (or
registers in the case of load double) are stored in \code{loadReg1}
and \code{loadReg2} (if this is a load double).  The corresponding
values to be stored in these registers (on the next cycle) are stored
in \code{loadValue1} and \code{loadValue2}.

When an instruction that reads registers (such as an \code{ADD}
instruction) is encountered during simulation, the contents of
\code{loadReg1} and \code{loadReg2} are examined before any other
action occurs.  If either of the registers specified by these
variables were loaded in the previous instruction, a load stall is
detected and tallied.  Different register fields must be checked for
different instructions.  All the load stall detection logic is
contained in the macros at the top of the \code{Simulate} function
definition.

Of interest is the fact that while load stalls would slow down the execution
speed of a real DLX machine, they do not affect the performance of the
simulator.  This is because load stall cycles are not actually simulated.
Instead, it is simply noted that a load stall occurred at a particular point, and
execution proceeds normally.

\subsubsection{Dynamic Instruction Counts}

Statistics on the number of each type of instruction executed are also 
recorded during simulation.  This is a simple operation of incrementing the
appropriate element of the array \code{operationCount}, which is indexed by
the pseudo-opcodes discussed above.  The information in the array can be
accessed by the {\tt stats} command.

\subsubsection{Conditional Branch Behavior}

\dlxsim\ also keeps statistics on the conditional branch behavior during
program execution.  There are four instructions in this category: 
\code{BEQZ}, \code{BNEZ}, \code{BFPT}, and \code{BFPF}.  The latter two
instructions are branches based on the status of the floating point
condition register.  Two fields of the DLX machine structure, \code{branchYes}
and \code{branchNo} record how many conditional branches where taken and
not taken, respectively.  These values are accessible via the {\tt stats}
command.

\subsection{Floating Point Execution Control}

A large portion of the \dlxsim\ code is devoted to the floating point side
of the machine.  The floating point scheme currently implemented requires
instructions to issue in order, but they may complete out of order.  In addition
to managing the allocation of the floating point units,
\dlxsim\ must also handle all the hazard checking associated with out of order
completion of instructions.  By requiring instructions to issue in order, the
write-after-read (WAR) hazard is avoided.  The three hazards which may occur
are read-after-write (RAW) hazards, write-after-write (WAW) hazards, and
structural hazards.

\subsubsection{Floating Point Data Structures}

The variables and data structures which manage the floating point
execution are all declared in the file \code{dlx.h} as part of the basic
DLX structure.  The variables \code{num\_add\_units}, \code{num\_div\_units},
and \code{num\_mul\_units} specify how many of each type of floating point
execution unit are available on the machine.  
The variables \code{fp\_add\_latency}, \code{fp\_div\_latency}, and 
\code{fp\_mul\_latency} specify the corresponding latencies (in clock cycles)
of each of the execution units.  All six of these variables have default
values which may be overridden via command line parameters when
\dlxsim\ is invoked.

The variable \code{FPstatusReg} is the status register which is examined
on a \code{BFPT} or \code{BFPF} instruction.  The various floating point
set instructions (\code{EQF}, \code{NED}, etc.) write to this register.

The array \code{fp\_add\_units} contains the status of all the floating point
adders during execution.  If \code{fp\_add\_units[i]} is zero, adder i is
available.  A non-zero value means that the unit is currently performing an
operation -- the value specifies the clock cycle when the operation will 
complete.  The array \code{fp\_div\_units} and \code{fp\_mul\_units} contain
analogous information for the floating point dividers and multipliers.  All
three structures can be accessed through the array \code{fp\_units} which is
an array of pointers to the three execution unit status arrays.

The array \code{waiting\_FPRs} contains 32 elements, corresponding to the
32 floating point registers in DLX.  A zero in \code{waiting\_FPRs[i]} means
floating point register Fi can be read from; it contains its most current
value.  A non-zero value means register Fi is the destination register of a
pending floating point operation (one which has issued but not yet completed).
Attempting to read or write to such a register means a hazard condition exists.
The non-zero value indicates the cycle at which the writeback to the register
will occur.

The variable \code{FPopsList} points to the chain of pending floating point
operations.  Each item in this chain is of type \code{FPop}, a structure
with the following fields:
\begin{mylist}
  \item [\code{type} \hfill] Indicates the type of operation.  Normally this is
   implied by what type of floating point unit is executing the operation, 
   however adders can perform both additions and subtractions.
  \item [\code{unit} \hfill] The unit number of the execution unit which is
   executing the operation.
  \item [\code{dest} \hfill] The destination register for the operation.  For a
   double precision operation, this is the lower-numbered destination register.
  \item[\code{isDouble} \hfill] Indicates if the operation is single or double
   precision.
  \item [\code{result} \hfill] An array of two floats used to store the result of the
   operation (only the first element is used for single precision operations).
   The result is actually computed at the time of issue.
  \item [\code{ready} \hfill] The cycle when the operation will complete and writeback
   will occur.
  \item[\code{nextPtr} \hfill] Points to the next \code{FPop} in the chain of
   pending operations.
\end{mylist}
To maximize performance, the list of pending floating point operations is
sorted based on when the operations will complete.  The operation which
will complete soonest is at the head of the list.

The variable \code{checkFP} is a copy of the \code{ready} field of the
first floating point operation on the pending operation list.  If its
value is zero, no floating point operations are pending.  Otherwise 
\code{checkFP} indicates when the next (soonest) floating point operation will
complete.  This provides for very quick checking in the fast-path of the
simulator.  Only one value needs to be checked in a cycle when no
writebacks should occur.

Many of the previously
discussed  structures refer to a clock cycle count when a particular
operation will complete.  The current clock cycle is kept in the
variable \code{cycleCount}.  This variable is incremented each time the
simulator executes its main loop.  It is also incremented an
extra time when a load stall is detected since the floating point units are
still executing during a load stall.  When the cycle count reaches a large value
specified by the constant \code{CYC\_CNT\_RESET}, \code{cycleCount} is ``reset''
back to a small number (5), and all references to clock cycles in the
floating point data structures are adjusted accordingly.  This operation is
necessary to prevent \code{cycleCount} from overflowing, becoming negative, and
thereby wreaking havoc on the sorted list of pending operations.  Making
\code{cycleCount} an unsigned integer does not work, since there are still
problems with sorting the pending operations when cycle counts ``wrap around''
to zero.

\subsubsection{Issuing Floating Point Operations}

The function \code{FPissue} initiates a floating point operation.  It is called
from eight of the switch cases in the main loop:  \code{ADDF}, \code{DIVF},
\code{MULF}, \code{SUBF},  \code{ADDD}, \code{DIVD}, \code{MULD}, 
and \code{SUBD}.  When a floating point instruction issues, three hazard
conditions must be checked.  A structural hazard occurs if a floating point
unit of the required type is not available.  A RAW hazard occurs if one of
the source operands is the destination of a pending floating point operation.
Finally, a WAW hazard occurs if the destination register is the destination
register of a pending floating point operation.  All three conditions can
be checked by examing the floating point data structures discussed above.  If
any of these hazards are present (and there may be more than one), the
current instruction is not issued.  Instead a non-zero value is returned
which indicates the soonest cycle when one of the hazard conditions will
be over.  This may be a cycle when one of the floating point units will 
complete its current operation (eliminating a structural hazard), or
when some register will be written back (eliminating a RAW or WAW hazard).
When the caller receives a non-zero value from \code{FPissue}, the appropriate
number of floating point stalls are simulated by adjusting the variables
\code{cycleCount} and \code{FPstalls}.  The function \code{FPwriteBack}
(see below) is called to perform any writebacks which may now occur.  Then
\code{FPissue} is re-invoked.  If another hazard condition exists, the whole
process may be repeated, but eventually all of the hazard conditions will
terminate.

If no hazards are present, the instruction is issued.  That is, an new \code{FPop}
structure is placed in the appropriate spot in the pending operations list.
The appropriate elements of \code{waiting\_FPRs} are also set to indicate
that the destination registers are waiting for values to be written back.
\code{FPissue} returns a zero value to indicate a successful issue, and the
simulation continues.

\subsubsection{Writing Back Floating Point Results}

The function \code{FPwriteBack} is the second function involved in
floating point execution.  It is called whenever \code{cycleCount} reaches
\code{checkFP}, indicating that a result is ready to be written back on the
current cycle.  \code{FPwriteBack} does exactly that.  It removes the
first \code{FPop} from the list of pending operations, and stores the
result (computed at time of issue) in the appropriate register(s).  It also
zeroes the appropriate element(s) in \code{waiting\_FPRs}.  Since more than
one operation may complete on the same cycle, \code{FPwriteBack} repeats
this process until the value in the \code{ready} field of the operation
at the head of the list exceeds the current value in \code{cycleCount}.

\subsubsection{Handling RAW and WAW Hazards}

The function \code{FPissue} (discussed above) handles the RAW and WAW 
hazards when a new floating point operation is issued.  However, several other
instructions can generate such hazards.  Any instruction which reads from or
writes to a floating point register must check that the register is not the
destination of a pending operation.  The following instructions fall into
this class:
\begin{mylist}
  \item[Loads \hfill]  LF and LD.
  \item[Stores \hfill] SF and SD.
  \item[Moves \hfill] MOVFP2I, MOVI2FP, MOVF, MOVD.
  \item[Converts \hfill] CVTD2FP, CVTD2I, CVTFP2D, CVTFP2I, CVTI2D, CVTI2FP.
  \item[Sets \hfill] SEQF, SNEF, SLTF, SLEF, SGTF, SGEF, SEQD, SNED, SLTD, SLED, 
   SGTD, SGED.
\end{mylist}
When any of these instruction are executed, a call to \code{FPwait} is made.
This is the third and final function for handling floating point execution.
It checks that all writebacks into the appropriate registers have occurred.
The number of registers which need to be checked varies.  For a LF 
instruction, only a single register needs to be checked, while four registers
must be checked on a MOVD.  If any of the registers are the destinations
of pending operations, \code{FPwait} will adjust \code{cycleCount} and
\code{FPstalls} appropriately, and call \code{FPwriteBack} to write the
results back to the registers.  When \code{FPwait} returns, all RAW and WAW
hazard conditions will have passed.


\end{document}
